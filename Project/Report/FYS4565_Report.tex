\documentclass[12pt,a4paper,english]{article}
\usepackage{times}
\usepackage[utf8]{inputenc}
\usepackage{babel,textcomp}
\usepackage{mathpazo}
\usepackage{mathtools}
\usepackage{amsmath,amssymb}
\usepackage{ dsfont }
\usepackage{listings}
\usepackage{graphicx}
\usepackage{float}
\usepackage{subfig} 
\usepackage[colorlinks]{hyperref}
\usepackage[usenames,dvipsnames,svgnames,table]{xcolor}
\usepackage{textcomp}
\definecolor{listinggray}{gray}{0.9}
\definecolor{lbcolor}{rgb}{0.9,0.9,0.9}
\lstset{backgroundcolor=\color{lbcolor},tabsize=5,rulecolor=,language=matlab,basicstyle=\scriptsize,upquote=true,aboveskip={1.5\baselineskip},columns=fixed,numbers=none,showstringspaces=false,extendedchars=false,breaklines=true,
prebreak=\raisebox{0ex}[0ex][0ex]{\ensuremath{\hookleftarrow}},frame=single,showtabs=false,showspaces=false,showstringspaces=false,identifierstyle=\ttfamily,keywordstyle=\color[rgb]{0.9,0.9,0.9},commentstyle=\color[rgb]{0.133,0.545,0.133},stringstyle=\color[rgb]{0.627,0.126,0.941},literate={å}{{\r a}}1 {Å}{{\r A}}1 {ø}{{\o}}1}

% Use for references
%\usepackage[sort&compress,square,comma,numbers]{natbib}
%\DeclareRobustCommand{\citeext}[1]{\citeauthor{#1}~\cite{#1}}

% Fix spacing in tables and figures
%\usepackage[belowskip=-8pt,aboveskip=5pt]{caption}
%\setlength{\intextsep}{10pt plus 2pt minus 2pt}

% Change the page layout
%\usepackage[showframe]{geometry}
\usepackage{layout}
\setlength{\hoffset}{-0.4in}  % Length left
%\setlength{\voffset}{-1.1in}  % Length on top
\setlength{\textwidth}{460pt}  % Width /597pt
%\setlength{\textheight}{720pt}  % Height /845pt
%\setlength{\footskip}{25pt}

\newcommand{\VEV}[1]{\langle#1\rangle}
\title{FYS4565 - Obligatory exercise:\\ Emittance preservation techniques in a linear accelerator}
\date{}
\author{ Kristoffer Langstad\\ \textit{krilangs@uio.no}}

\begin{document}%\layout
\maketitle
\section*{Exercises}
\label{sect:Exercises}
\subsection*{Theory}
\label{subsect:Theory}
\subsubsection*{1)}
The (geometric) emittance, $\varepsilon$, is defined as a space area. The normalized emittance, $\varepsilon$, is defined as the emittance which is preserved while the beam is accelerated when there are source of emittance growth. This normalized emittance is given as:
\begin{equation}
\label{eq:norm_em}
\varepsilon_N=\beta\gamma\varepsilon
\end{equation}
$\beta$ and $\gamma$ are the relativistic functions normalized velocity and the Lorentz factor, respectively. Geometric emittance is only constant when there is no acceleration involved with the beam, while normalized emittance does not change due to acceleration.

\subsubsection*{2)}
For a beam with N number of particles and phase space ${x_n,x_n^{\prime}}$, the Twiss parameters are given as:
\begin{align}
\beta &= \frac{<x_n^2>}{\varepsilon}\\
\alpha &= -\frac{1}{2}\beta=-\frac{<x_nx_n^{\prime}>}{\varepsilon}\\
\gamma &= \frac{1+\alpha^2}{\beta}=\frac{<{x_n^{\prime}}^2>}{\varepsilon}
\end{align} 
The geometric emittance is given as:
\begin{equation}
\varepsilon= \sqrt{<x_n^2><{x_n^{\prime}}^2>-<x_nx_n^{\prime}>^2}
\end{equation}

\subsection*{Emittance growth}
\label{subsect:Emittance}
\subsubsection*{1)}
Run script and track a beam with \textbf{runMADX()} gives the following output:
\begin{lstlisting}
Running MADX...   ++++++++++++++++++++++++++++++++++++++++++++
+     MAD-X 5.05.02  (64 bit, Windows)     +
+ Support: mad@cern.ch, http://cern.ch/mad +
+ Release   date: 2019.07.25               +
+ Execution date: 2020.04.08 15:37:12      +
++++++++++++++++++++++++++++++++++++++++++++
OPTION, ECHO = FALSE, TWISS_PRINT = FALSE, INFO = FALSE;

initial->betx      =        3.410060363 ;
initial->bety      =        0.590010444 ;
initial->alfx      =       -2.404092556 ;
initial->alfy      =        0.415957363 ;
emit_x             =          5.11e-009 ;
emit_y             =          5.11e-009 ;

GXPLOT-X11  1.50 initialized 

plot number =            1 
plot number =            2 
enter TRACK module
one pass is on
exit TRACK module


Number of warnings: 0

++++++++++++++++++++++++++++++++++++++++++++
+          MAD-X finished normally         +
++++++++++++++++++++++++++++++++++++++++++++
Done.
\end{lstlisting}
This is to show that the program works as wanted.

\subsubsection*{2)}
Load the initial beam characterized by the 6D phase space distribution ($x,x^{\prime},y,y^{\prime}, z, \Delta E/E$). The three first values for each of the 6D phase space parameters are printed to show the program works:
\begin{lstlisting}
Initial beam:
-3.372303e-05 9.786400e-05 1.370667e-04 -5.637760e-06 1.168475e-04 9.629966e-05 
-3.942910e-05 8.925327e-05 1.275999e-04 -3.026930e-05 6.234654e-05 -5.830864e-05 
3.634007e-06 -1.178952e-05 1.350098e-06 -3.673267e-03 -4.841775e-03 1.045882e-02
\end{lstlisting}

\subsubsection*{3)}
Calculate the rms normalized emittance of the input beam from th phase space distribution. The geometric emittance is calculated to: 
\[\varepsilon_x=5.108212e-09\quad \varepsilon_y=5.105969e-09\]
The normalized emittance is calculated to:
\[\varepsilon_{N,x}=9.996502e-06\quad \varepsilon_{N,y}=9.992112e-06\]
This is the answers that we expected to get as they coincide with the input parameters ($10\mu m$).

\subsubsection*{4)}
The Twiss parameters for the input beam are calculated to give:
\begin{align*}
\alpha_x&=-2.416237,\quad \beta_x=3.414410\\
\alpha_y&=0.4237378,\quad \beta_y=0.5949598
\end{align*}
These calculated Twiss parameters are around the expected values of the MADX output of the FODO-matched solution with an expected relative error of around $1/\sqrt{N}$:
\begin{lstlisting}
initial->betx      =        3.410060363 ;
initial->bety      =        0.590010444 ;
initial->alfx      =       -2.404092556 ;
initial->alfy      =        0.415957363 ;
\end{lstlisting}

\subsubsection*{5)}
In Figure \ref{fig:beam_orbit} we see the beam position monitor (BPM) orbit ($\overline{x},\overline{y}$ vs. $s$). This BPM system determines the position of the beam, which is needed due to that the beam trajectory may be deflected from the pipe center.

\begin{figure}[htbp!]
	\centering\includegraphics[width=0.6\linewidth, height=0.5\linewidth]{orbit.jpg}
	\caption{BPM orbit: $\overline{x},\overline{y}$ vs. $s$ . \label{fig:beam_orbit}}
\end{figure} 

\subsubsection*{6)}
Introduce quadrupole misalignments with a 1mm rms offset in both directions. The new BPM orbit is seen in Figure \ref{fig:mis_beam_orbit}. The mean BPM positions vary a lot more when quadrupole misalignments are introduced compared to without.

\begin{figure}[htbp!]
	\centering\includegraphics[width=0.6\linewidth, height=0.5\linewidth]{orbit_mis.jpg}
	\caption{BPM orbit with a 1 mm rms offset quadrupole misalignment in both directions: $\overline{x},\overline{y}$ vs. $s$. \label{fig:mis_beam_orbit}}
\end{figure} 

\subsubsection*{7)}
The dispersion function, $D(s)$, of an accelerator beam line is defined as the local sensitivity of the beam trajectory, $x(s)$, to a relative energy error $\Delta E/E$:
\begin{equation}
D(s)=\frac{x(s)}{\Delta E/E}
\end{equation}
This is calculated by first tracking a beam with the nominal energy and then by tracking a beam with slightly different energy. In Figure \ref{fig:dispersion} we see a plot of this dispersion function difference between the two beams in both directions.

\begin{figure}[htbp!]
	\centering\includegraphics[width=0.6\linewidth, height=0.5\linewidth]{dispersion.jpg}
	\caption{Plot of the dispersion function as a function of the BPM s-position. \label{fig:dispersion}}
\end{figure} 

\subsubsection*{8)}
Here we set the energy spread of the input beam to zero. In Figure \ref{fig:em_growth} we see a plot of the relative emittance growth from the start of the lattice to the end of the lattice as a function of quadrupole misalignment.

\begin{figure}[htbp!]
	\centering\includegraphics[width=0.6\linewidth, height=0.5\linewidth]{growth.jpg}
	\caption{Relative emittance growth from the start of the lattice to the end of the lattice as a function of quadrupole misalignment. \label{fig:em_growth}}
\end{figure} 

\subsubsection*{9)}
Here we set the energy spread of the input beam to 1\%. In Figure \ref{fig:em_growth_spread} we see a plot of the relative emittance growth from the start of the lattice to the end of the lattice as a function of quadrupole misalignment with the small energy spread.

\begin{figure}[htbp!]
	\centering\includegraphics[width=0.6\linewidth, height=0.5\linewidth]{growth_spread.jpg}
	\caption{Relative emittance growth from the start of the lattice to the end of the lattice as a function of quadrupole misalignment with a small energy spread (1\%). \label{fig:em_growth_spread}}
\end{figure} 

\subsubsection*{10)}
For 0\% energy spread, the relative emittance growth does not change that much, increases up till only around $6\cdot10^{-7}$\%, in the lattice as the quadrupole misalignment increases. In the x-direction the emittance growth increases almost exponential, while in the y-direction it decreases a little. For 1\% energy spread, the relative emittance growth increases much more, up till almost 13\%, in the lattice as the quadrupole misalignment increases. Here the emittance growth in the x-direction is more linear, and now it increases also a little in the y-direction as well. The emittance can be seen as the area of the phase space of the position and momentum. So when there is a higher energy spread, the bigger the area becomes. This seems to be reasonable since when a particle has higher energy, it also has a higher momentum leading to a larger distance covered by the particle. So a larger energy spread should lead to a higher emittance.	

\subsubsection*{11)}
Change the quad strength $k$ of the FODO-cells by $\pm20$\%, and recalculate the emittance growth with a small energy spread as earlier. In Figure \ref{fig:quad_strength} we see how the emittance growth behaves in the lattice as function of quadrupole misalignment when we change the quad strength. There we see that as the quad strength increases, the higher is the emittance growth increase as the quadrupole misalignment increases. Increasing the quad strength will increase the energy on the beam giving the beam more momentum. As we have already seen, the more energy/momentum the beam has the higher is the emittance growth as function of quadrupole misalignments in the lattice.

\begin{figure}[htbp!]
	\centering\includegraphics[width=0.6\linewidth, height=0.5\linewidth]{growth_spread_k.jpg}
	\caption{Emittance growth as function of quadrupole misalignment as we change the quad strength and with a small energy spread. \label{fig:quad_strength}}
\end{figure} 

\subsection*{Beam-based correction}
\label{subsect:Beam correction}

\end{document}
